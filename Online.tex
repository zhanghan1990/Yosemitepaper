%\section{Main Idea of Cache Placement}

In the proposed system, the placement in each time-bin is decided based on the predicted arrival rates in the time bin. The time bin can either be a fixed time or dynamic based on significant change of the predicted arrival rates. At the start of the time-bin, new cache placement is found using the optimized algorithm. For each file that has lower number of chunks in the new time bin, the decreased contents are removed from the cache. For the files for which the cache contents increase in a time bin, we wait for the file to be accessed. When the file is accessed,  the file contents are gathered and the required new chunks are generated to be placed in the cache. Thus, the change of cache content do not place any additional network overhead and the cache contents of a file are added only when it is first accessed in the new time bin. This process can be further improved improving latency till convergence to the new cache content in the new time bin by not letting all the chunks which have to be removed all simultaneously but removing as needed based on the added chunks.